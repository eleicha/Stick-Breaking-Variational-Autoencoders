%!TEX root = paper.tex
\section{Introduction}
\label{intro}
% 
%The rise of the Internet of Things (IoT) has changed drastically the field of data management, due its challenging characteristics. 
%One major challenge is to manage
Data analytics applications for the Internet of Things (IoT), such as reporting and monitoring dashboards, consist of real-time data preprocessing and data mining tasks.
Such applications visualize high-velocity data streams, resulting from large sensor networks, which flow through heterogeneous hardware and network topologies to the cloud.

Since sensor data matches naturally the provided stream processing abstractions, today's IoT applications use cloud-based stream processing engines (SPEs) for their data management tasks. 
Furthermore, SPEs exploit the on-demand scalability of cloud-resources to support compute-intensive data management workloads.
However, state-of-the-art SPEs, such as Apache Flink~\cite{flink}, were designed for cloud environments composed of homogeneous high-performance hardware, where nodes are interconnected through high-speed network connections. \textcolor{red}{Laura: Ich finde, der Absatz stört den Lesefluss etwas. Können wir den vielleicht rausnehmen?}

In contrast, IoT infrastructures have different characteristics, both regarding the nodes and the network connections~\cite{fog}. \textcolor{red}{Laura: Hier fände ich den Absatz passender, da der letzte Satz einen guten Abschluss zum vorherigen Absatz bildet und hier dann quasi das "neue Konzept" erklärt wird.}
Sensor nodes that are potentially geographically distributed, continuously generate data, resulting in a large number of data streams with small-sized records. Intermediate nodes route the produced sensor data to the cloud.
In this new type of infrastructure, intermediate nodes are heterogeneous, geographically distributed, and sparsely interconnected through unstable networks. 
In particular, IoT devices range from low-end nodes, such as system-on-a-chip devices and cheap sensors, to high-end nodes, such as desktop computers and server racks. \textcolor{red}{Laura: Same as above.}

Hence, using cloud-based SPEs for IoT applications restricts scaling data management operations only within the cloud. \textcolor{red}{Laura: Same as above. Hence insbesondere bezieht sich hier noch stark auf den vorherigen Teil, weshalb ich es hier auf jeden Fall wichtig fände, den Absatz anders zu wählen.} To scale data management tasks on all participating devices of an IoT infrastructure, the design of data management systems must be revisited.
Recent work points out, that to exploit resources of every node in an IoT infrastructure, data management systems must employ infrastucture-aware execution strategies~\cite{nes}.

A data management system for the IoT should leverage the scale-out capabilities of the cloud, and at the same time exploit the resources of intermediate nodes. The set of intermediate nodes, also known as fog~\cite{fog}, are only utilized for data forwarding when using cloud-based systems for data management. In particular, the cloud can scale resources for compute-intensive tasks, while nodes in the fog can apply in-network processing~\cite{in-network}, fog computing techniques~\cite{frontier} and acquisitional data processing~\cite{tinydb} to reduce intermediate results. By reducing intermediate results, network traffic and cloud resources are minimized.

NebulaStream (NES)~\cite{nes} is an application and data management platform designed for the upcoming IoT era. 
NES addresses the mentioned IoT challenges by unifying sensor, fog, and cloud nodes into a single system.
NES combines research from sensor network, distributed and database system communities. 
This allows NES to transparently optimize and efficiently execute data management workloads across IoT infrastructures.

Overall, this unified cloud-fog approach enables scaling the number of sensors in data management and visualization applications, while efficiently exploiting the existing hardware infrastructure.
We demonstrate a visualization application for a public transport system on top of NES. 
Our application aims to provide real-time monitoring for public transport systems, and detects geographical areas that are critical, i.e., public transport underserves those areas.
Our GUI consists of an interactive map, which visualizes real-time public transport vehicles, and potential passengers. 
Through our GUI, visitors are able to move the map, filter vehicles, and configure parameters for the critical area detection. To detect underserved areas, we use a density-based clustering algorithm.
Public transport systems could use this information to reschedule vehicles manually, or even automatically.

Using the public transport of a city as a representative IoT application, we demonstrate how all participating nodes can be part of a query, and how this influences performance and resource utilization. 
Overall, we showcase that NES allows large-scale applications on IoT infrastructures and thus enables a variety of upcoming IoT application use cases.

The rest of this paper is structured as follows. In \Cref{iot}, we discuss challenges related to data management in IoT infrastructures. In \Cref{nes}, we provide a brief overview of NES's design principles and its architecture. After that, in \Cref{demo}, we present our demonstration scenario and finally  conclude in \Cref{conclusion}.

